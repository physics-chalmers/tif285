%%
%% Automatically generated file from DocOnce source
%% (https://github.com/hplgit/doconce/)
%%

% #define PREAMBLE

% #ifdef PREAMBLE
%-------------------- begin preamble ----------------------

\documentclass[%
oneside,                 % oneside: electronic viewing, twoside: printing
final,                   % draft: marks overfull hboxes, figures with paths
10pt]{article}

\listfiles               %  print all files needed to compile this document

\usepackage{relsize,makeidx,color,setspace,amsmath,amsfonts,amssymb}
\usepackage[table]{xcolor}
\usepackage{bm,ltablex,microtype}

\usepackage[pdftex]{graphicx}

\usepackage[T1]{fontenc}
%\usepackage[latin1]{inputenc}
\usepackage{ucs}
\usepackage[utf8x]{inputenc}

\usepackage{lmodern}         % Latin Modern fonts derived from Computer Modern

% Hyperlinks in PDF:
\definecolor{linkcolor}{rgb}{0,0,0.4}
\usepackage{hyperref}
\hypersetup{
    breaklinks=true,
    colorlinks=true,
    linkcolor=linkcolor,
    urlcolor=linkcolor,
    citecolor=black,
    filecolor=black,
    %filecolor=blue,
    pdfmenubar=true,
    pdftoolbar=true,
    bookmarksdepth=3   % Uncomment (and tweak) for PDF bookmarks with more levels than the TOC
    }
%\hyperbaseurl{}   % hyperlinks are relative to this root

\setcounter{tocdepth}{2}  % levels in table of contents

% Tricks for having figures close to where they are defined:
% 1. define less restrictive rules for where to put figures
\setcounter{topnumber}{2}
\setcounter{bottomnumber}{2}
\setcounter{totalnumber}{4}
\renewcommand{\topfraction}{0.95}
\renewcommand{\bottomfraction}{0.95}
\renewcommand{\textfraction}{0}
\renewcommand{\floatpagefraction}{0.75}
% floatpagefraction must always be less than topfraction!
% 2. ensure all figures are flushed before next section
\usepackage[section]{placeins}
% 3. enable begin{figure}[H] (often leads to ugly pagebreaks)
%\usepackage{float}\restylefloat{figure}

% --- fancyhdr package for fancy headers ---
\usepackage{fancyhdr}
\fancyhf{} % sets both header and footer to nothing
\renewcommand{\headrulewidth}{0pt}
\fancyfoot[LE,RO]{\thepage}
% Ensure copyright on titlepage (article style) and chapter pages (book style)
\fancypagestyle{plain}{
  \fancyhf{}
  \fancyfoot[C]{{\footnotesize \copyright\ 2018-2019, Christian Forssén. Released under CC Attribution-NonCommercial 4.0 license}}
%  \renewcommand{\footrulewidth}{0mm}
  \renewcommand{\headrulewidth}{0mm}
}
% Ensure copyright on titlepages with \thispagestyle{empty}
\fancypagestyle{empty}{
  \fancyhf{}
  \fancyfoot[C]{{\footnotesize \copyright\ 2018-2019, Christian Forssén. Released under CC Attribution-NonCommercial 4.0 license}}
  \renewcommand{\footrulewidth}{0mm}
  \renewcommand{\headrulewidth}{0mm}
}

\pagestyle{fancy}


% prevent orhpans and widows
\clubpenalty = 10000
\widowpenalty = 10000

% --- end of standard preamble for documents ---


\usepackage[swedish]{babel}

\raggedbottom
\makeindex
\usepackage[totoc]{idxlayout}   % for index in the toc
\usepackage[nottoc]{tocbibind}  % for references/bibliography in the toc

%-------------------- end preamble ----------------------

\begin{document}

% matching end for #ifdef PREAMBLE
% #endif

\newcommand{\exercisesection}[1]{\subsection*{#1}}

\input{newcommands_keep}

% ------------------- main content ----------------------



% ----------------- title -------------------------

\thispagestyle{empty}

\begin{center}
{\LARGE\bf
\begin{spacing}{1.25}
Learning from data: Convolutional Neural Networks
\end{spacing}
}
\end{center}

% ----------------- author(s) -------------------------

\begin{center}
{\bf Christian Forssén${}^{1}$} \\ [0mm]
\end{center}


\begin{center}
{\bf Morten Hjorth-Jensen${}^{2, 3}$} \\ [0mm]
\end{center}

\begin{center}
% List of all institutions:
\centerline{{\small ${}^1$Department of Physics, Chalmers University of Technology, Sweden}}
\centerline{{\small ${}^2$Department of Physics, University of Oslo}}
\centerline{{\small ${}^3$Department of Physics and Astronomy and National Superconducting Cyclotron Laboratory, Michigan State University}}
\end{center}
    
% ----------------- end author(s) -------------------------

% --- begin date ---
\begin{center}
Oct 20, 2019
\end{center}
% --- end date ---

\vspace{1cm}


% !split
\section{Convolutional Neural Networks}

Convolutional Neural Networks (CNNs) are very similar to ordinary Neural Networks, but are particularly adopted for image recognition.

\begin{itemize}
\item They are made up of layers that have learnable weights and biases. 

\item The inputs are operated on with dot products, typically followed by a non-linear activation function. 

\item The whole network still expresses a single differentiable score function: from the raw image pixels on one end to class scores at the other. 

\item And they still have a loss function (for example Softmax) on the last (fully-connected) layer. 

\item Learning takes place via back propagation, gradient descent, etc.
\end{itemize}

\noindent
What is the difference? \textbf{CNN architectures make the explicit assumption that
the inputs are images, which allows us to encode certain properties
into the architecture. These then make the forward function more
efficient to implement and vastly reduce the amount of parameters in
the network.}

Here we provide only a superficial overview.

% !split
\subsection{Regular NNs don’t scale well to full images}

As an example, consider
an image of size $32\times 32\times 3$ (32 wide, 32 high, 3 color channels), so a
single fully-connected neuron in a first hidden layer of a regular
Neural Network would have $32\times 32\times 3 = 3072$ weights. This amount still
seems manageable, but clearly this fully-connected structure does not
scale to larger images. For example, an image of more respectable
size, say $200\times 200\times 3$, would lead to neurons that have 
$200\times 200\times 3 = 120,000$ weights. 

We could have several such neurons, and the parameters would add up quickly! Clearly,
this full connectivity is wasteful and the huge number of parameters
would quickly lead to possible overfitting.


\begin{figure}[!ht]  % 
  \centerline{\includegraphics[width=0.6\linewidth]{fig/nn.jpeg}}
  \caption{
  A regular 3-layer Neural Network.
  }
\end{figure}
%\clearpage % flush figures 


% !split
\subsection{3D volumes of neurons}

CNNs take advantage of the fact that the
input consists of images and they constrain the architecture in a more
sensible way. 

In particular, unlike a regular Neural Network, the
layers of a CNN have neurons arranged in 3 dimensions: width,
height, depth. (Note that the word depth here refers to the third
dimension of an activation volume, not to the depth of a full Neural
Network, which can refer to the total number of layers in a network.)

To understand it better, the above example of an image 
with an input volume of
activations has dimensions $32\times 32\times 3$ (width, height,
depth respectively). 

The neurons in a layer will
only be connected to a small region of the layer before it, instead of
all of the neurons in a fully-connected manner. Moreover, the final
output layer could  for this specific image have dimensions $1\times 1 \times 10$, 
because by the
end of the CNN architecture we will reduce the full image into a
single vector of class scores, arranged along the depth
dimension. 


\begin{figure}[!ht]  % 
  \centerline{\includegraphics[width=0.6\linewidth]{fig/cnn.jpeg}}
  \caption{
  A CNN arranges its neurons in three dimensions (width, height, depth), as visualized in one of the layers. Every layer of a CNN transforms the 3D input volume to a 3D output volume of neuron activations. In this example, the red input layer holds the image, so its width and height would be the dimensions of the image, and the depth would be 3 (Red, Green, Blue channels).
  }
\end{figure}
%\clearpage % flush figures 




% !split 
\subsection{Layers used to build CNNs}

A simple CNN is a sequence of layers, and every layer of a CNN
transforms one volume of activations to another through a
differentiable function. We use three main types of layers to build
CNN architectures: Convolutional Layer, Pooling Layer, and
Fully-Connected Layer (exactly as seen in regular Neural Networks). We
will stack these layers to form a full CNN architecture.

% !split 
The layers of a convolutional neural network arrange neurons in 3D: width, height and depth.  
The input image is typically a square matrix of depth 3. 

A \textbf{convolution} is performed on the image which outputs
a 3D volume of neurons. The weights to the input are arranged in a number of 2D matrices, known as \textbf{filters}.

Each filter slides along the input image, taking the dot product
between each small part of the image and the filter, in all depth
dimensions. This is then passed through a non-linear function,
typically the \textbf{Rectified Linear (ReLu)} function, which serves as the
activation of the neurons in the first convolutional layer. This is
further passed through a \textbf{pooling layer}, which reduces the size of the
convolutional layer, e.g.~by taking the maximum or average across some
small regions, and this serves as input to the next convolutional
layer.

\paragraph{Example: CNN architecture.}
A simple CNN for image classification could have the architecture:

\begin{itemize}
\item \textbf{INPUT} ($32\times 32 \times 3$) will hold the raw pixel values of the image, in this case an image of width 32, height 32, and with three color channels R,G,B.

\item \textbf{CONV} (convolutional )layer will compute the output of neurons that are connected to local regions in the input, each computing a dot product between their weights and a small region they are connected to in the input volume. This may result in volume such as $[32\times 32\times 12]$ if we decided to use 12 filters.

\item \textbf{RELU} layer will apply an elementwise activation function, such as the $max(0,x)$ thresholding at zero. This leaves the size of the volume unchanged ($[32\times 32\times 12]$).

\item \textbf{POOL} (pooling) layer will perform a downsampling operation along the spatial dimensions (width, height), resulting in volume such as $[16\times 16\times 12]$.

\item \textbf{FC} (i.e.~fully-connected) layer will compute the class scores, resulting in volume of size $[1\times 1\times 10]$, where each of the 10 numbers correspond to a class score, such as among the 10 categories of the MNIST images we considered above . As with ordinary Neural Networks and as the name implies, each neuron in this layer will be connected to all the numbers in the previous volume.
\end{itemize}

\noindent
% !split
\paragraph{Systematic reduction.}
By systematically reducing the size of the input volume, through
convolution and pooling, the network should create representations of
small parts of the input, and then from them assemble representations
of larger areas.  The final pooling layer is flattened to serve as
input to a hidden layer, such that each neuron in the final pooling
layer is connected to every single neuron in the hidden layer. This
then serves as input to the output layer, e.g.~a softmax output for
classification.

% !split
\subsection{Transforming images}

CNNs transform the original image layer by layer from the original
pixel values to the final class scores. 

Observe that some layers contain
parameters and other don’t. In particular, the CNN layers perform
transformations that are a function of not only the activations in the
input volume, but also of the parameters (the weights and biases of
the neurons). On the other hand, the RELU/POOL layers will implement a
fixed function. The parameters in the CONV/FC layers will be trained
with gradient descent so that the class scores that the CNN computes
are consistent with the labels in the training set for each image.

% !split
\paragraph{Example: The MNIST dataset.}
The MNIST dataset consists of grayscale images with a pixel size of
$28\times 28$, meaning we require $28 \times 28 = 724$ weights to each
neuron in the first hidden layer.

If we were to analyze images of size $128\times 128$ we would require
$128 \times 128 = 16384$ weights to each neuron. Even worse if we were
dealing with color images, as most images are, we have an image matrix
of size $128\times 128$ for each color dimension (Red, Green, Blue),
meaning 3 times the number of weights $= 49152$ are required for every
single neuron in the first hidden layer.

% !split
\paragraph{Setting it up.}
It means that to represent the entire
dataset of images, we require a 4D matrix or \textbf{tensor}. This tensor has the dimensions:  
\[  
(n_\mathrm{inputs},\, n_\mathrm{pixels, width},\, n_\mathrm{pixels, height},\, n_\mathrm{pixels, depth}) .
\]

% !split
\subsection{CNNs in brief}

In summary:

\begin{itemize}
\item A CNN architecture is in the simplest case a list of layers that transform the image volume into an output volume (e.g.~holding the class scores)

\item There are a few distinct types of layers (e.g.~CONV/FC/RELU/POOL)

\item Each layer accepts an input 3D volume and transforms it to an output 3D volume through a differentiable function

\item Each layer may or may not have parameters (e.g.~CONV/FC do, RELU/POOL don’t)

\item Each layer may or may not have additional hyperparameters (e.g.~CONV/FC/POOL do, RELU doesn’t)
\end{itemize}

\noindent
For more material on convolutional networks, we strongly recommend
the slides of \href{{http://cs231n.github.io/convolutional-networks/}}{CS231} which is taught at Stanford University. Furthermore, Michael Nielsen's book \href{{http://neuralnetworksanddeeplearning.com/index.html}}{Neural Networks and Deep Learning} is a very good read, in particular chapter 6 which deals with \href{{http://neuralnetworksanddeeplearning.com/chap6.html}}{CNNs}.

% ------------------- end of main content ---------------

% #ifdef PREAMBLE
\end{document}
% #endif

